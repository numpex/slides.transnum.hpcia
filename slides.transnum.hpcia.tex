\documentclass[aspectratio=169]{beamer}

\usetheme{numpex}

\usepackage{tikz,pgfplots}
\usepackage{booktabs}
\usepackage[scale=2]{ccicons}

\usepackage{tikz,pgfplots}
\usepackage{booktabs}
\usetikzlibrary{positioning,shapes.geometric,arrows.meta}


\title{HPC-AI convergence}
\subtitle{AI-in-HPC, HPC-for-AI, AI-as-Oracle}
\date{\today}
\author{J Bobin, J. Gusak, C. Prud'homme, J.P. Vilotte}
\institute{PEPR NumPEx}

\begin{document}

\maketitle

\begin{frame}{Why This Workshop? Success Criteria}

\textbf{Context and Stakes}
\begin{itemize}
  \item \textbf{NumPEx post-exascale:} HPC--AI convergence is inevitable, but pathways are multiple
  \item \textbf{Audience:} computer scientists, HPC mathematicians, decision-makers, ...
  \item \textbf{Horizon:} 10 years, milestones \textbf{2028} and \textbf{2030}
\end{itemize}

\vspace{0.3cm}
\textbf{Success Criteria (end of workshop)}
\begin{itemize}
  \item A \textbf{shared vision} (X--Y key messages) for CSA positioning
  \item A \textbf{roadmap} with concrete outcomes for 2028 and 2030
  \item \textbf{Sandboxes} projects to test ideas and build expertise
  \item Open \textbf{Guidelines} for our community and beyond
\end{itemize}

\note[item]{Remind participants: 2h30 total across two sessions (Tuesday + Wednesday)}
\note[item]{This material will serve the CSA (Scientific and Administrative Council)}
\note[item]{Ask: "What must we have produced by end of workshop to consider it successful?"}
\end{frame}

\begin{frame}{The 3 Topics of HPC--AI Convergence}

\begin{center}
\begin{tikzpicture}[
  topic/.style={rectangle, rounded corners, draw=nblue, fill=nblue!10, thick, minimum width=5cm, minimum height=1.2cm, align=center, font=\small},
  node distance=0.6cm
]
  \node[topic] (T1) {\textbf{T1: AI in HPC}\\(hybrid simulation + AI components)};
  \node[topic, below=of T1] (T2) {\textbf{T2: AI as Oracle}\\(agentic AI, dev, perf, tuning)};
  \node[topic, below=of T2] (T3) {\textbf{T3: HPC for AI}\\(training/inference at scale, co-design)};
\end{tikzpicture}
\end{center}

\vspace{0.2cm}
\textbf{Each topic is interrogated by:}
\begin{itemize}
  \item Cross-cutting requirements (trust, energy, sovereignty, skills, benchmarks)
  \item Flagship use-case: \alert{"DT operations + AI advisor"} (stress-tests all 3 topics)
\end{itemize}

\note[item]{T1 = hybrid models (surrogates, closures, inversions, neural operators)}
\note[item]{T2 = agentic AI for dev, debug, optim (e.g., GitHub Copilot for HPC)}
\note[item]{T3 = HPC infrastructure to train/infer LLMs and foundation models}
\note[item]{We'll revisit each topic with cross-cutting requirements and the flagship use-case}
\end{frame}

\begin{frame}{Cross-Cutting Requirements --- Checklist for T1--T3}

\textbf{Apply systematically to each topic:}
\begin{itemize}
  \item \textbf{Trust \& certification:} V\&V, UQ, reproducibility, traceability. What do we certify?
  \item \textbf{Energy \& resource constraints:} J/solution vs J/insight, CO₂/experiment. Frugal post-exascale.
  \item \textbf{Software sustainability:} AI vs HPC pace, qualification, portability of stacks.
  \item \textbf{Sovereignty:} build vs buy, lock-in, critical dependencies.
  \item \textbf{Roles \& skills:} HPC software eng, scientific MLOps, data steward, performance engineer.
  \item \textbf{Benchmarking \& reporting:} metrics, eval suites (prompts/tests for agentic HPC tools).
  \item \textbf{New technologies:} Quantum, Neuromorphic arch, Photonic in transit Computing
\end{itemize}

%\vspace{0.2cm}
\note[item]{Ask participants: "For each topic, which checklist items are blockers?"}
\note[item]{Emphasize: post-exascale = energy-constrained, we can't do everything}
\note[item]{Sovereignty: avoid single-vendor cloud or proprietary framework lock-in}
\note[item]{Skills: which new profiles to hire? Which training programs?}
\end{frame}

\begin{frame}{Flagship Use-Case: DT Operations + AI Advisor}

\textbf{Why this use-case as a filter?}
\begin{itemize}
  \item \textbf{Operational Digital Twin:} real-time or near-real-time simulations (latency-critical)
  \item \textbf{AI advisor:} suggests parameters, detects anomalies, optimizes online
  \item \textbf{Stress-tests all 3 topics simultaneously}
\end{itemize}

\vspace{0.4cm}
\begin{columns}[T]
\column{0.32\textwidth}
\alert{\textbf{T1: AI in HPC}}
\begin{itemize}\small
  \item Fast surrogates
  \item ML closures
  \item Guarantees, UQ
  \item Out-of-domain detection
\end{itemize}

\column{0.32\textwidth}
\alert{\textbf{T2: AI as Oracle}}
\begin{itemize}\small
  \item Orchestration
  \item Observability
  \item Governance, audit
  \item Trace exploitation
\end{itemize}

\column{0.32\textwidth}
\alert{\textbf{T3: HPC for AI}}
\begin{itemize}\small
  \item Model updates
  \item Training at scale
  \item Data generation
  \item Drift monitoring
\end{itemize}
\end{columns}

\note[item]{Operational DT must respond in seconds/minutes, not hours}
\note[item]{AI advisor = system proposing actions (change process param, trigger maintenance)}
\note[item]{This use-case combines all 3 topics: needs infrastructure (T3), surrogates (T1), agentic dev tools (T2)}
\note[item]{Ask: "Who has similar use-cases? What are your pain points?"}
\end{frame}

\begin{frame}{DT + AI Advisor: Operationalization Requirements}

\textbf{This flagship use-case reveals critical requirements:}

\vspace{0.3cm}
\begin{itemize}
  \item \textbf{Lifecycle management:} versioning (model+data), drift monitoring, retraining triggers, rollback
  \item \textbf{Auditability \& trust:} logs, provenance, responsibility, explainability
  \item \textbf{Degraded modes:} safe fallback when AI is uncertain, default behaviors
  \item \textbf{SLA/KPIs:} time-to-insight, energy-to-insight, confidence levels
\end{itemize}

\vspace{0.4cm}
\textbf{These requirements apply beyond DT:}
\begin{itemize}
  \item Online steering and optimization
  \item Continuous data assimilation
  \item Automated decision workflows
  \item Any production AI-in-HPC system
\end{itemize}

\note[item]{Emphasize: operationalization is not an afterthought, it's a design requirement}
\note[item]{These requirements apply to any "AI in production" scenario in HPC context}
\note[item]{Ask: "Which of these requirements are you addressing in your projects?"}
\end{frame}

\begin{frame}{AI-as-Oracle}
    
\textbf{What's the future of code development at the AI-dominated era ?}

\vspace{0.3cm}
\begin{itemize}
  \item \textbf{Trust \& certification:} AI will likely take a large part of the inner loop of software development (code generation/translation/porting - orchestration/optimisation/automation), where the cursor should be put between AI-assisted to AI-automated developement ? How to verify (consistency) and validate (domain-specific performance) AI-generated codes ?
  \item \textbf{Data and qualification:} HPC codes are increasingly complex (e.g. heterogenous hardware), can AI be trusted for code generation at all levels of the software stack ? How to ensure the the qualification of AI-generated/orchestrated HPC codes ?
\end{itemize}

 % \textbf{Apply systematically to each topic:}
%\begin{itemize}
 % \item \textbf{Trust \& certification:} V\&V, UQ, reproducibility, %traceability. What do we certify?
 %\item \textbf{Energy \& resource constraints:} J/solution vs %J/insight, CO₂/experiment. Frugal post-exascale.
 % \item \textbf{Software sustainability:} AI vs HPC pace, %qualification, portability of stacks.
  %\item \textbf{Sovereignty:} build vs buy, lock-in, critical %dependencies.
  %\item \textbf{Roles \& skills:} HPC software eng, scientific MLOps, %data steward, performance engineer.
  %\item \textbf{Benchmarking \& reporting:} metrics, eval suites %(prompts/tests for agentic HPC tools).
  %\item \textbf{New technologies:} Quantum, Neuromorphic arch, %Photonic in transit Computing
%\end{itemize}
\end{frame}

  
\begin{frame}{AI-as-Oracle}
    
\textbf{What's the future of code development at the AI-dominated era ?}

\vspace{0.3cm}
\begin{itemize}
    \item \textbf{Skills :} what skills future computer scientists should have ? Promp-engineers only ?\\
    
    \item \textbf{Energy \& resource constraints:} AI has a significant energy cost. How to optimize the energy cost of AI for software development ? \\

    \item \textbf{Data:} \& how to ensure data availability (e.g. from simulations to traces/log) and trust, which are central for validation and robustness ?
    
\end{itemize}

\end{frame}


\begin{frame}{AI-in-HPC}
  
\end{frame}


\begin{frame}{HPC-for-AI}
  
\end{frame}

\end{document}


%%% Local Variables:
%%% mode: latex
%%% TeX-master: t
%%% TeX-engine: xetex
%%% TeX-command-extra-options: "--synctex=1"
%%% End:
