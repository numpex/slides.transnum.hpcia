\documentclass[final]{beamer}
\usepackage[size=a0,scale=1.4]{beamerposter}
\usetheme{numpex}
\usepackage{tikz}
\usepackage{graphicx}
\usepackage{xcolor}
\usepackage{pgfplots}
\usepackage{booktabs}
\usepackage{url}
\usepackage{listings}
\lstset{
  basicstyle=\ttfamily\scriptsize, % Set the font and size for code listings
  breaklines=true,                  % Automatically break lines
  breakatwhitespace=true,          % Break lines at whitespace
  numbers=left,                    % Add line numbers on the left
  numberstyle=\tiny,               % Set the style for line numbers
  frame=single,                   % Add a frame around the code
  showspaces=false,               % Do not show spaces
  showstringspaces=false,         % Do not show spaces in strings
  showtabs=false,                 % Do not show tabs
  tabsize=2                       % Set the tab size
}


% 1) Set a poster background
\setbeamertemplate{background}{
  \begin{tikzpicture}[remember picture,overlay]
    \node at (current page.center)      {\includegraphics[height=\paperheight, keepaspectratio=false]{images/NumPEx_1.pdf}};
  \end{tikzpicture}
}


% 2) Override block colors for a white background
\setbeamercolor{block title}{bg=white, fg=blue!60!black}
\setbeamercolor{block body}{bg=white, fg=black}

% 3) Headline with a larger white rectangle (5cm high) and updated logos (3cm high)
\setbeamertemplate{headline}{
  \begin{tikzpicture}[remember picture,overlay]
    % White headline background (5cm tall)
    \fill[white] (current page.north west) rectangle ([yshift=-5cm] current page.north east);

    % Left logos
    \node[anchor=north west, inner sep=0.5cm] at (current page.north west)
      {%
        \includegraphics[height=3cm]{images/rf.png}\hspace{0.3cm}%
        \includegraphics[height=3cm]{images/f2030.png}%
      };

    % Right logos
    \node[anchor=north east, inner sep=0.5cm] at (current page.north east)
      {%
        \includegraphics[height=3cm]{images/cea.jpg}\hspace{0.3cm}%
        \includegraphics[height=3cm]{images/inria.png}\hspace{0.3cm}%
        \includegraphics[height=3cm]{images/cnrs.png}%
      };

    % Title (centered in the white area)
    \node[anchor=south, yshift=0.5cm] at ([yshift=-5cm] current page.north)
    {\parbox{\paperwidth}{\centering
      {\usebeamerfont{title}\color{black}\inserttitle\par\vspace{0.3cm}
       {\usebeamerfont{author}\color{black}\insertauthor\par}}
    }};
  \end{tikzpicture}
}

% 4) Footline: a white rectangle (2cm tall) with logos and centered text
\setbeamertemplate{footline}{
  \begin{tikzpicture}[remember picture,overlay]
    % White footline background (2cm tall)
    \fill[white] (current page.south west) rectangle ([yshift=2cm] current page.south east);

    % Left logos in footline (scaled to 1.5cm height)
    \node[anchor=south west, inner sep=0.5cm] at (current page.south west)
      {%
        \includegraphics[height=1.5cm]{images/rf.png}\hspace{0.2cm}%
        \includegraphics[height=1.5cm]{images/f2030.png}%
      };

    % Right logos in footline (scaled to 1.5cm height)
    \node[anchor=south east, inner sep=0.5cm] at (current page.south east)
      {%
        \includegraphics[height=1.5cm]{images/cea.jpg}\hspace{0.2cm}%
        \includegraphics[height=1.5cm]{images/inria.png}\hspace{0.2cm}%
        \includegraphics[height=1.5cm]{images/cnrs.png}%
      };

    % Centered footline text (e.g. institute and date)
    \node[anchor=south] at ([yshift=0.5cm] current page.south)
      {\parbox{\paperwidth}{\centering
      {\usebeamerfont{footline}\usebeamercolor[fg]{footline}\insertinstitute \quad \insertdate}}};
  \end{tikzpicture}
}
\setbeamercolor{block title}{ use=normal text,  bg=purple!15!white, fg=black }
\setbeamertemplate{block begin}{
  \begin{beamercolorbox}[wd=\linewidth,sep=5pt,shadow=true,rounded=true]{block title}%
    \usebeamerfont{block title}\insertblocktitle
  \end{beamercolorbox}%
  \begin{beamercolorbox}[wd=\linewidth,sep=5pt,rounded=true]{block body}%
    \usebeamerfont{block body}%
}
\setbeamertemplate{block end}{
  \end{beamercolorbox}
}


\title{A Poster Beamer Theme for NumPEx}
\author{Alfredo Buttari, Christophe Prud'homme}
\institute{Institute or Miscellaneous Information}
\date{\today}

\begin{document}

\begin{frame}[t,fragile]{}
  % Leave space for the headline (5cm tall)
  \vspace*{5cm}
  \begin{center}
    % Use a centered minipage 90% of the poster width
    \begin{minipage}{0.9\paperwidth}
      \begin{columns}[t,totalwidth=\textwidth]

        % Column 1
        \column{0.32\textwidth}
          \begin{block}{Theme Overview}
            The NumPEx theme is a Beamer theme with minimal visual noise inspired by the HSRM theme. It emphasizes a clean design with elegant typography (using the Marianne font, so XeTeX is required).
            \vspace{0.3cm}
            \textbf{Enable the theme by loading:}
            \begin{verbatim}
\usetheme{numpex}
            \end{verbatim}
            \vspace{0.3cm}
            See the French state graphic guidelines:
            \begin{center}
              \alert{\href{https://www.info.gouv.fr/marque-de-letat}{Charte graphique de l'Etat Français}}
            \end{center}
          \end{block}

          \vspace{0.5cm}

          \begin{block}{Sections \& Typography}
            Sections group slides on the same topic. For example:
            \begin{verbatim}
\section{Elements}
            \end{verbatim}
            The theme provides sensible defaults for:
            \begin{itemize}
              \item \emph{Emphasis} on key text.
              \item \alert{Accents} on important parts.
              \item \textbf{Bold} for results.
              \item Also \textbf{\emph{combine}} the previous ones.
            \end{itemize}
          \end{block}

        % Column 2
        \column{0.32\textwidth}
          \begin{block}{Lists, Descriptions \& Animation}
            \textbf{Lists:}
            \begin{itemize}
              \item Milk, Eggs, Potatoes.
            \end{itemize}
            \textbf{Enumerations:}
            \begin{enumerate}
              \item First, \item Second, \item Last.
            \end{enumerate}
            \textbf{Descriptions:}
            \begin{description}
              \item[PowerPoint] Meeh.
              \item[Beamer] Yeeeha.
            \end{description}
            \textbf{Animation:}
            \begin{itemize}
              \item \alert{This is really important.}
              \item Now this.
              \item And now this.
            \end{itemize}
          \end{block}

          \vspace{0.5cm}

          \begin{block}{Tables}
            \begin{table}
              \caption{Largest cities in the world (source: Wikipedia)}
              \begin{tabular}{lr}
                \toprule
                City & Population\\
                \midrule
                Mexico City & 20,116,842\\
                Shanghai & 19,210,000\\
                Peking & 15,796,450\\
                Istanbul & 14,160,467\\
                \bottomrule
              \end{tabular}
            \end{table}
          \end{block}

        % Column 3
        \column{0.32\textwidth}
          \begin{block}{Blocks \& Math}
            \begin{block}{Simple Block}
              Hello
            \end{block}

            \vspace{0.3cm}
            \textbf{Mathematics:}
            \[
              e = \lim_{n\to \infty} {\left(1 + \frac{1}{n}\right)}^n
            \]
          \end{block}
          \vspace{0.5cm}
          \begin{exampleblock}{Example Block}
            Hello
          \end{exampleblock}

          \vspace{0.5cm}
          \begin{block}{Line Plots \& Bar Charts}
            \textbf{Line Plot:}
            \begin{center}
              \begin{tikzpicture}[domain=0:4, scale=1.5, every node/.style={scale=0.7}]
                \draw[very thin,color=gray] (-0.1,-1.1) grid (3.9,3.9);
                \draw[->] (-0.2,0) -- (4.2,0) node[right, font=\scriptsize] {$x$};
                \draw[->] (0,-1.2) -- (0,4.2) node[above, font=\scriptsize] {$f(x)$};
                \draw[color=red] plot (\x,\x) node[right, font=\scriptsize] {$f(x)=x$};
                \draw[color=blue] plot (\x,{sin(\x r)}) node[right, font=\scriptsize] {$f(x)=\sin x$};
                \draw[color=orange] plot (\x,{0.05*exp(\x)}) node[right, font=\scriptsize] {$f(x)=\frac{1}{20}e^x$};
              \end{tikzpicture}
            \end{center}
            \vspace{0.3cm}
            \textbf{Bar Chart:}
            \begin{center}
              \begin{tikzpicture}[scale=1.5, every node/.style={scale=0.8}]
                \begin{axis}[
                  x tick label style={/pgf/number format/1000 sep=},
                  ylabel={\scriptsize Population},
                  enlargelimits=0.15,
                  legend style={at={(0.5,-0.15)},anchor=north,legend columns=-1, font=\scriptsize},
                  ybar,
                  bar width=5pt,
                  tick label style={font=\scriptsize},
                  label style={font=\scriptsize},
                ]
                  \addplot [color=red, fill=red!30] coordinates {(1930,50e6) (1940,33e6) (1950,40e6) (1960,50e6) (1970,70e6)};
                  \addplot [color=blue, fill=blue!30] coordinates {(1930,38e6) (1940,42e6) (1950,43e6) (1960,45e6) (1970,65e6)};
                  \addplot [color=green, fill=green!30] coordinates {(1930,15e6) (1940,12e6) (1950,13e6) (1960,25e6) (1970,35e6)};
                  \addplot[red,sharp plot,update limits=false] coordinates {(1910,4.3e7) (1990,4.3e7)}
                    node[above, font=\scriptsize] at (axis cs:1950,4.3e7) {Houses};
                  \legend{{\scriptsize Far},{\scriptsize Near},{\scriptsize Here},{\scriptsize Annot}}
                \end{axis}
              \end{tikzpicture}
            \end{center}

          \end{block}

          \vspace{0.5cm}

          \begin{exampleblock}{Code Example}
            \begin{lstlisting}[basicstyle=\tt\scriptsize, showlines=true]
              program pippo
                integer, parameter :: m=10, n=5
                real(kind(1.d0)), allocatable :: a(:,:)
                allocate(a(m,n))
                call random_number(a)
                call do_something(a)
                write(*,'("Hello world")')
                stop
              end program pippo
                          \end{lstlisting}
          \end{exampleblock}

          \begin{block}{Conclusion}
            Get the source of this theme and the demo presentation from:
            \begin{center}\small
              \url{https://github.com/numpex/presentation.template/}
            \end{center}
          \end{block}

      \end{columns}
    \end{minipage}
  \end{center}
\end{frame}

\end{document}
%%% Local Variables:
%%% mode: latex
%%% TeX-master: t
%%% TeX-engine: xetex
%%% TeX-command-extra-options: "--synctex=1"
%%% End:
